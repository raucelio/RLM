% Options for packages loaded elsewhere
\PassOptionsToPackage{unicode}{hyperref}
\PassOptionsToPackage{hyphens}{url}
%
\documentclass[
]{book}
\usepackage{lmodern}
\usepackage{amssymb,amsmath}
\usepackage{ifxetex,ifluatex}
\ifnum 0\ifxetex 1\fi\ifluatex 1\fi=0 % if pdftex
  \usepackage[T1]{fontenc}
  \usepackage[utf8]{inputenc}
  \usepackage{textcomp} % provide euro and other symbols
\else % if luatex or xetex
  \usepackage{unicode-math}
  \defaultfontfeatures{Scale=MatchLowercase}
  \defaultfontfeatures[\rmfamily]{Ligatures=TeX,Scale=1}
\fi
% Use upquote if available, for straight quotes in verbatim environments
\IfFileExists{upquote.sty}{\usepackage{upquote}}{}
\IfFileExists{microtype.sty}{% use microtype if available
  \usepackage[]{microtype}
  \UseMicrotypeSet[protrusion]{basicmath} % disable protrusion for tt fonts
}{}
\makeatletter
\@ifundefined{KOMAClassName}{% if non-KOMA class
  \IfFileExists{parskip.sty}{%
    \usepackage{parskip}
  }{% else
    \setlength{\parindent}{0pt}
    \setlength{\parskip}{6pt plus 2pt minus 1pt}}
}{% if KOMA class
  \KOMAoptions{parskip=half}}
\makeatother
\usepackage{xcolor}
\IfFileExists{xurl.sty}{\usepackage{xurl}}{} % add URL line breaks if available
\IfFileExists{bookmark.sty}{\usepackage{bookmark}}{\usepackage{hyperref}}
\hypersetup{
  pdftitle={Regressão Linear Múltipla},
  hidelinks,
  pdfcreator={LaTeX via pandoc}}
\urlstyle{same} % disable monospaced font for URLs
\usepackage{color}
\usepackage{fancyvrb}
\newcommand{\VerbBar}{|}
\newcommand{\VERB}{\Verb[commandchars=\\\{\}]}
\DefineVerbatimEnvironment{Highlighting}{Verbatim}{commandchars=\\\{\}}
% Add ',fontsize=\small' for more characters per line
\usepackage{framed}
\definecolor{shadecolor}{RGB}{248,248,248}
\newenvironment{Shaded}{\begin{snugshade}}{\end{snugshade}}
\newcommand{\AlertTok}[1]{\textcolor[rgb]{0.94,0.16,0.16}{#1}}
\newcommand{\AnnotationTok}[1]{\textcolor[rgb]{0.56,0.35,0.01}{\textbf{\textit{#1}}}}
\newcommand{\AttributeTok}[1]{\textcolor[rgb]{0.77,0.63,0.00}{#1}}
\newcommand{\BaseNTok}[1]{\textcolor[rgb]{0.00,0.00,0.81}{#1}}
\newcommand{\BuiltInTok}[1]{#1}
\newcommand{\CharTok}[1]{\textcolor[rgb]{0.31,0.60,0.02}{#1}}
\newcommand{\CommentTok}[1]{\textcolor[rgb]{0.56,0.35,0.01}{\textit{#1}}}
\newcommand{\CommentVarTok}[1]{\textcolor[rgb]{0.56,0.35,0.01}{\textbf{\textit{#1}}}}
\newcommand{\ConstantTok}[1]{\textcolor[rgb]{0.00,0.00,0.00}{#1}}
\newcommand{\ControlFlowTok}[1]{\textcolor[rgb]{0.13,0.29,0.53}{\textbf{#1}}}
\newcommand{\DataTypeTok}[1]{\textcolor[rgb]{0.13,0.29,0.53}{#1}}
\newcommand{\DecValTok}[1]{\textcolor[rgb]{0.00,0.00,0.81}{#1}}
\newcommand{\DocumentationTok}[1]{\textcolor[rgb]{0.56,0.35,0.01}{\textbf{\textit{#1}}}}
\newcommand{\ErrorTok}[1]{\textcolor[rgb]{0.64,0.00,0.00}{\textbf{#1}}}
\newcommand{\ExtensionTok}[1]{#1}
\newcommand{\FloatTok}[1]{\textcolor[rgb]{0.00,0.00,0.81}{#1}}
\newcommand{\FunctionTok}[1]{\textcolor[rgb]{0.00,0.00,0.00}{#1}}
\newcommand{\ImportTok}[1]{#1}
\newcommand{\InformationTok}[1]{\textcolor[rgb]{0.56,0.35,0.01}{\textbf{\textit{#1}}}}
\newcommand{\KeywordTok}[1]{\textcolor[rgb]{0.13,0.29,0.53}{\textbf{#1}}}
\newcommand{\NormalTok}[1]{#1}
\newcommand{\OperatorTok}[1]{\textcolor[rgb]{0.81,0.36,0.00}{\textbf{#1}}}
\newcommand{\OtherTok}[1]{\textcolor[rgb]{0.56,0.35,0.01}{#1}}
\newcommand{\PreprocessorTok}[1]{\textcolor[rgb]{0.56,0.35,0.01}{\textit{#1}}}
\newcommand{\RegionMarkerTok}[1]{#1}
\newcommand{\SpecialCharTok}[1]{\textcolor[rgb]{0.00,0.00,0.00}{#1}}
\newcommand{\SpecialStringTok}[1]{\textcolor[rgb]{0.31,0.60,0.02}{#1}}
\newcommand{\StringTok}[1]{\textcolor[rgb]{0.31,0.60,0.02}{#1}}
\newcommand{\VariableTok}[1]{\textcolor[rgb]{0.00,0.00,0.00}{#1}}
\newcommand{\VerbatimStringTok}[1]{\textcolor[rgb]{0.31,0.60,0.02}{#1}}
\newcommand{\WarningTok}[1]{\textcolor[rgb]{0.56,0.35,0.01}{\textbf{\textit{#1}}}}
\usepackage{longtable,booktabs}
% Correct order of tables after \paragraph or \subparagraph
\usepackage{etoolbox}
\makeatletter
\patchcmd\longtable{\par}{\if@noskipsec\mbox{}\fi\par}{}{}
\makeatother
% Allow footnotes in longtable head/foot
\IfFileExists{footnotehyper.sty}{\usepackage{footnotehyper}}{\usepackage{footnote}}
\makesavenoteenv{longtable}
\usepackage{graphicx}
\makeatletter
\def\maxwidth{\ifdim\Gin@nat@width>\linewidth\linewidth\else\Gin@nat@width\fi}
\def\maxheight{\ifdim\Gin@nat@height>\textheight\textheight\else\Gin@nat@height\fi}
\makeatother
% Scale images if necessary, so that they will not overflow the page
% margins by default, and it is still possible to overwrite the defaults
% using explicit options in \includegraphics[width, height, ...]{}
\setkeys{Gin}{width=\maxwidth,height=\maxheight,keepaspectratio}
% Set default figure placement to htbp
\makeatletter
\def\fps@figure{htbp}
\makeatother
\setlength{\emergencystretch}{3em} % prevent overfull lines
\providecommand{\tightlist}{%
  \setlength{\itemsep}{0pt}\setlength{\parskip}{0pt}}
\setcounter{secnumdepth}{5}
\usepackage{booktabs}
\usepackage[]{natbib}
\bibliographystyle{apalike}

\title{Regressão Linear Múltipla}
\author{}
\date{\vspace{-2.5em}}

\begin{document}
\maketitle

{
\setcounter{tocdepth}{1}
\tableofcontents
}
\hypertarget{introduuxe7uxe3o}{%
\chapter{Introdução}\label{introduuxe7uxe3o}}

\hypertarget{intro}{%
\chapter{Introdução a matrizes}\label{intro}}

O estudo de Regressão Linear Múltipla requer conhecimento de notação matricial. Seguem as definições necessárias para o início do estudo desta técnica.

Neste texto são apresentadas definições e operações utilizadas na regressão linear múltipla, já a operação com matrizes no R será apresentada em outro capítulo.

\hypertarget{definiuxe7uxf5es}{%
\section{Definições}\label{definiuxe7uxf5es}}

\begin{itemize}
\tightlist
\item
  \textbf{Matriz} é um arranjo retangular de elementos organizados em linhas e em colunas. Uma matriz é denotada por uma letra maíuscula e os elementos por letras minúsculas, como em A:
\end{itemize}

\begin{equation*}
A = 
\begin{bmatrix}
a_{11} & a_{12} & a_{13} & \cdots & a_{1n} \\
a_{21} & a_{22} & a_{23} & \cdots & a_{2n} \\
\vdots &        &        & \cdots &        \\
a_{m1} & a_{m2} & a_{m3} & \cdots & a_{mn} \\
\end{bmatrix}
\end{equation*}

Uma matriz com \(m \times n\) elementos, ordenados em \(m\) linhas e \(n\) colunas, é uma matriz de ordem \(m\) por \(n\) e denotada por \(m \times n\). Na notação \(a_{ij}\) o índice \(i\) indica a linha e o \(j\) a coluna.

\textbf{Exemplo}

\begin{equation*}
A = 
\begin{bmatrix}
2 & 5 & 10 \\
3 & 6 & 12
\end{bmatrix}
\end{equation*}

\begin{itemize}
\tightlist
\item
  \textbf{Matriz Quadrada} é caracterizada por ter o número de linhas igual ao numero de colunas, \(m = n\).
\end{itemize}

\textbf{Exemplo}

\begin{equation*}
A = 
\begin{bmatrix}
8 & 2 \\
4 & 10
\end{bmatrix}
\end{equation*}

Em uma matriz quadrada, os elementos \(a_{ij}\), com \(i = j\), forma a \textbf{diagonal principal}.

\begin{itemize}
\tightlist
\item
  \textbf{Matriz diagonal} é caracterizada por ter os elementos \(a_{ij} = 0\) para todo \(i \neq j\).
\end{itemize}

\textbf{Exemplo}

\begin{equation*}
A = 
\begin{bmatrix}
8 & 0 & 0\\
0 & 2 & 0\\
0 & 0 & 5
\end{bmatrix}
\end{equation*}

\begin{itemize}
\tightlist
\item
  \textbf{Matriz identidade} é uma matriz diagonal, cujos elementos da diagonal principal são iguais a 1. É donotada por \textbf{I}.
\end{itemize}

\textbf{Exemplo}

\begin{equation*}
I = 
\begin{bmatrix}
1 & 0 & 0\\
0 & 1 & 0\\
0 & 0 & 1
\end{bmatrix}
\end{equation*}

Observe que \(a_{ij} =0\) para \(i \neq j\) e \(a_{ij} = 1\) para \(i = j\).

\begin{itemize}
\tightlist
\item
  \textbf{Matriz nula} é uma matriz cujos elementos são iguais a 0.
\end{itemize}

\textbf{Exemplo}

\begin{equation*}
I = 
\begin{bmatrix}
0 & 0 & 0\\
0 & 0 & 0
\end{bmatrix}
\end{equation*}

Observe que \(a_{ij} =0\) para \(i \neq j\) e \(a_{ij} = 1\) para \(i = j\).

\begin{itemize}
\tightlist
\item
  \textbf{Matriz transposta} é o resultado da troca das linhas pelas colunas e denotada por \(A^T\) ou \(A^\prime\).
\end{itemize}

\textbf{Exemplo}

Para

\begin{equation*}
A = 
\begin{bmatrix}
2 & 7 \\
0 & 1 \\
2 & 0 
\end{bmatrix}
\end{equation*}

a sua transposta \(A^\prime\) é

\begin{equation*}
A^\prime = 
\begin{bmatrix}
2 & 0 & 2\\
7 & 1 & 0
\end{bmatrix}
\end{equation*}

\begin{itemize}
\tightlist
\item
  \textbf{Matriz Simétrica} é uma \textbf{matriz quadrada} com a propriedade de ser igual a sua transposta. Assim, \(A = A^\prime\).
\end{itemize}

\textbf{Exemplo}

\begin{equation*}
A = 
\begin{bmatrix}
1 & 2 & 3\\
2 & 1 & 4\\
3 & 4 & 1
\end{bmatrix}
\end{equation*}

\begin{itemize}
\tightlist
\item
  \textbf{Traço de matriz quadrada} é a soma dos elementos da diagonal principal:
\end{itemize}

\begin{equation*}
Traço(A)= \sum_{i,j} a_{ij} \quad \forall i = j
\end{equation*}

\textbf{Exemplo}

Para

\begin{equation*}
A = 
\begin{bmatrix}
7 & 2 & 3\\
2 & 8 & 4\\
3 & 4 & 9
\end{bmatrix}
\end{equation*}

os elementos da \textbf{diagonal principal} são 7, 8 e 9, logo:

\begin{equation*}
Traço(A)= 7 + 8 + 9 = 24
\end{equation*}

\begin{itemize}
\item
  \textbf{Determinante de matriz} é um valor associado a uma matriz quadrada, A, e denotada por \(|A|\).
\item
  \textbf{Matriz singular} é a matriz A cujo determinante é nulo, \(|A| = 0\).
\end{itemize}

\hypertarget{operauxe7uxf5es-com-matrizes}{%
\section{Operações com matrizes}\label{operauxe7uxf5es-com-matrizes}}

\begin{itemize}
\tightlist
\item
  \textbf{Soma e subtração} de duas matrizes \textbf{com as mesmas dimensões} podem ser somadas ou subtraídas adicionandos pela soma ou subtração dos elementos correspondentes.
\end{itemize}

\textbf{Exemplo}

\begin{equation*}
\begin{bmatrix}
7 & 2 & 3\\
2 & 8 & 4\\
3 & 4 & 9
\end{bmatrix} + 
\begin{bmatrix}
3 & 8 & 7\\
8 & 2 & 6\\
7 & 6 & 1
\end{bmatrix} = 
\begin{bmatrix}
10 & 10 & 10\\
10 & 10 & 10\\
10 & 10 & 10
\end{bmatrix}
\end{equation*}

\begin{itemize}
\tightlist
\item
  \textbf{Produto por escalar} de uma matriz A por um escalar c é obtida multiplicando cada elemento de A pelo valor de c.~
\end{itemize}

\textbf{Exemplo}

\begin{equation*}
5 \times
\begin{bmatrix}
3 & 8 & 7\\
8 & 2 & 6\\
7 & 6 & 1
\end{bmatrix} = 
\begin{bmatrix}
15 & 40 & 35\\
40 & 10 & 30\\
35 & 30 & 5
\end{bmatrix}
\end{equation*}

\begin{itemize}
\tightlist
\item
  \textbf{Produto de matrizes} só pode ser realizada caso o número de colunas da matriz que pré-multiplica for igual ao número de linhas da matriz que pós-multiplica. A operação consiste na soma dos produtos de cada elementos da linha da matriz que pré-multiplica pelos respectivos elementos da coluna da matriz que pós-multiplica. O resultado será uma matriz com o número de linhas da que pré-multiplica e com número de colunas das que pós-multiplica.
\end{itemize}

\textbf{Exemplo}

\begin{equation*}
\begin{bmatrix}
7 & 2 \\
2 & 8 \\
\end{bmatrix} + 
\begin{bmatrix}
3 & 8 & 7\\
8 & 2 & 6
\end{bmatrix} = 
\begin{bmatrix}
7 \times 3 + 2 \times 8 & 7 \times 8 + 2 \times 2 & 7 \times 7 + 2 \times 6\\
2 \times 3 + 8 \times 8 & 2 \times 8 + 8 \times 8 & 2 \times 7 + 8 \times 6
\end{bmatrix} =
\begin{bmatrix}
37 & 60 & 61\\
70 & 80 & 62
\end{bmatrix}
\end{equation*}

\begin{itemize}
\tightlist
\item
  \textbf{Matriz inversa} de uma matriz \(A\) é representada por \(A^{-1}\) e aquela que
\end{itemize}

\begin{equation*}
AA^{-1}=A^{-1}A = I
\end{equation*}

onde \(I\) é a matriz identidade. Para uma matriz \(A\) ter inversa é necessário e suficiente que seja \textbf{quadrada} e \textbf{não sigular}, isto é, o determinante é diferente de zero.

\textbf{Exemplo}

Para

\begin{equation*}
A = 
\begin{bmatrix}
2 & 1 & 1\\
1 & 3 & 1\\
0 & 1 & 2
\end{bmatrix}
\end{equation*}

a sua matriz inversa é

\begin{equation*}
A^{-1} = \dfrac{1}{9}
\begin{bmatrix}
 5 & -1 & -2\\
-2 &  4 & -1\\
 1 & -2 &  5
\end{bmatrix}
\end{equation*}

Verificando \(A^{-1}A=I\)

\begin{equation*}
\begin{bmatrix}
2 & 1 & 1\\
1 & 3 & 1\\
0 & 1 & 2
\end{bmatrix}
\dfrac{1}{9}
\begin{bmatrix}
 5 & -1 & -2\\
-2 &  4 & -1\\
 1 & -2 &  5
\end{bmatrix}=
\begin{bmatrix}
1 & 0 & 0\\
0 & 1 & 0\\
0 & 0 & 1
\end{bmatrix}
\end{equation*}

\hypertarget{propriedades-de-matrizes}{%
\section{Propriedades de matrizes}\label{propriedades-de-matrizes}}

Considerando que as operação de multiplicação, transposição e inversão das matrizes A, B e C. São válidas as seguintes propriedades.

\textbf{Multiplicação}

\begin{itemize}
\tightlist
\item
  \(ABC=A(BC)=A(BC)\)
\item
  \(A(B+C) = AB + AC\)
\item
  \((B+C)A = BA + CA\)
\end{itemize}

\textbf{Transposta de Matrizes}

\begin{itemize}
\tightlist
\item
  \(\left(A^\prime\right)^\prime = A\)
\item
  \(\left(A+B\right)^\prime= A^\prime + B^\prime\)
\item
  \(\left(AB\right)^\prime = B^\prime A^\prime\)
\end{itemize}

\textbf{Inversão de Matrizes}

\begin{itemize}
\tightlist
\item
  \(\left(A^{-1}\right)^{-1} = A\)
\item
  \(\left(AB\right)^{-1}= B^{-1}a^{-1}\)
\item
  \(\left(A^\prime\right)^{-1} = \left(A^{-1}\right)^\prime\)
\end{itemize}

\hypertarget{cap3}{%
\chapter{Matrizes no R}\label{cap3}}

O R tem uma estrutura de dados que organiza os seus valores em linhas e colunas, cujos os elemetentos devem ser do mesmo tipo. Essa estrutura é uma matriz e quando os seus valores são numéricos o seu coportamento é igual ao de uma matriz.

Uma \textbf{matriz} é um arranjo retangular de elementos organizados em linhas e em colunas. Uma matriz é denotada por uma letra maíuscula e os elementos por letras minúsculas, como em A:

\begin{equation*}
A = 
\begin{bmatrix}
a_{11} & a_{12} & a_{13} & \cdots & a_{1n} \\
a_{21} & a_{22} & a_{23} & \cdots & a_{2n} \\
\vdots &        &        & \cdots &        \\
a_{m1} & a_{m2} & a_{m3} & \cdots & a_{mn} \\
\end{bmatrix}
\end{equation*}

Uma matriz com \(m \times n\) elementos, ordenados em \(m\) linhas e \(n\) colunas, é uma matriz de ordem \(m\) por \(n\) e denotada por \(m \times n\). Na notação \(a_{ij}\) o índice \(i\) indica a linha e o \(j\) a coluna.

\hypertarget{gerar-matrizes-no-r}{%
\section{Gerar matrizes no R}\label{gerar-matrizes-no-r}}

A função \texttt{matrix()} gera uma matriz com \(m \times n\) valores organizados em \(m\) linhas e \(n\) colunas. A matriz

\begin{equation*}
A = 
\begin{bmatrix}
2 & 5 & 10 \\
3 & 6 & 12
\end{bmatrix}
\end{equation*}

pode ser gerada pelo código abaixo.

\begin{Shaded}
\begin{Highlighting}[]
\NormalTok{dados \textless{}{-}}\StringTok{ }\KeywordTok{c}\NormalTok{(}\DecValTok{2}\NormalTok{,}\DecValTok{3}\NormalTok{,}\DecValTok{5}\NormalTok{,}\DecValTok{6}\NormalTok{,}\DecValTok{10}\NormalTok{,}\DecValTok{12}\NormalTok{)}
\NormalTok{m =}\StringTok{ }\DecValTok{2} \CommentTok{\# Número de linhas}
\NormalTok{n =}\StringTok{ }\DecValTok{3} \CommentTok{\# Número de colunas }
\NormalTok{A \textless{}{-}}\StringTok{ }\KeywordTok{matrix}\NormalTok{(dados, }\DataTypeTok{nrow =}\NormalTok{ m, }\DataTypeTok{ncol =}\NormalTok{ n)}
\NormalTok{A}
\end{Highlighting}
\end{Shaded}

\begin{verbatim}
##      [,1] [,2] [,3]
## [1,]    2    5   10
## [2,]    3    6   12
\end{verbatim}

Observe que a matriz gerada tem \(m=2\) linhas e \(n=3\) colunas e os \(n\times m = 6\) valores são alocados por colunas, iniciando pela 1ª coluna da 1ª linha, esse comporpamento é o padrão e pode ser alterado pelo argumento \texttt{byrow\ =T}.

Uma \textbf{matriz quadrada} é caracterizada por ter o número de linhas igual ao numero de colunas, \(m = n\). Assim,

\begin{equation*}
A = 
\begin{bmatrix}
8 & 2 \\
4 & 10
\end{bmatrix}
\end{equation*}

A matriz acima é gerada por

\begin{Shaded}
\begin{Highlighting}[]
\NormalTok{dados \textless{}{-}}\StringTok{ }\KeywordTok{c}\NormalTok{(}\DecValTok{8}\NormalTok{,}\DecValTok{4}\NormalTok{,}\DecValTok{2}\NormalTok{,}\DecValTok{10}\NormalTok{)}
\NormalTok{m =}\StringTok{ }\DecValTok{2} \CommentTok{\# Número de linhas}
\NormalTok{n =}\StringTok{ }\DecValTok{2} \CommentTok{\# Número de colunas }
\NormalTok{A \textless{}{-}}\StringTok{ }\KeywordTok{matrix}\NormalTok{(dados, }\DataTypeTok{nrow =}\NormalTok{ m, }\DataTypeTok{ncol =}\NormalTok{ n)}
\NormalTok{A}
\end{Highlighting}
\end{Shaded}

\begin{verbatim}
##      [,1] [,2]
## [1,]    8    2
## [2,]    4   10
\end{verbatim}

Em uma matriz quadrada, os elementos \(a_{ij}\), com \(i = j\), forma a \textbf{diagonal principal}. Os elementos de uma diagonal podem ser obtidos pela função \texttt{diag()}.

\begin{Shaded}
\begin{Highlighting}[]
\NormalTok{A \textless{}{-}}\StringTok{ }\KeywordTok{matrix}\NormalTok{(}\KeywordTok{c}\NormalTok{(}\DecValTok{2}\NormalTok{,}\DecValTok{5}\NormalTok{,}\DecValTok{5}\NormalTok{,}\DecValTok{3}\NormalTok{),}\DecValTok{2}\NormalTok{,}\DecValTok{2}\NormalTok{)}
\KeywordTok{diag}\NormalTok{(A)}
\end{Highlighting}
\end{Shaded}

\begin{verbatim}
## [1] 2 3
\end{verbatim}

Uma matriz diagonal é caracterizada por ter os elementos \(a_{ij} = 0\) para todo \(i \neq j\). A função \texttt{diag()}, além de obter os elementos da diagonal principal, gera uma matriz diagonal. A matriz

\begin{equation*}
A = 
\begin{bmatrix}
8 & 0 & 0\\
0 & 2 & 0\\
0 & 0 & 5
\end{bmatrix}
\end{equation*}

pode ser criada pelo código

\begin{Shaded}
\begin{Highlighting}[]
\NormalTok{aux \textless{}{-}}\StringTok{ }\KeywordTok{c}\NormalTok{(}\DecValTok{8}\NormalTok{,}\DecValTok{2}\NormalTok{,}\DecValTok{5}\NormalTok{)}
\NormalTok{A\textless{}{-}}\StringTok{ }\KeywordTok{diag}\NormalTok{(aux)}
\NormalTok{A}
\end{Highlighting}
\end{Shaded}

\begin{verbatim}
##      [,1] [,2] [,3]
## [1,]    8    0    0
## [2,]    0    2    0
## [3,]    0    0    5
\end{verbatim}

A \textbf{Matriz identidade} pode ser uma matriz diagonal, cujos elementos da diagonal principal são iguais a 1. A matriz

\begin{equation*}
I = 
\begin{bmatrix}
1 & 0 & 0\\
0 & 1 & 0\\
0 & 0 & 1
\end{bmatrix}
\end{equation*}

pode ser gerada por

\begin{Shaded}
\begin{Highlighting}[]
\NormalTok{aux \textless{}{-}}\StringTok{ }\KeywordTok{rep}\NormalTok{(}\DecValTok{1}\NormalTok{,}\DecValTok{3}\NormalTok{)}
\NormalTok{I \textless{}{-}}\StringTok{ }\KeywordTok{diag}\NormalTok{(aux)}
\NormalTok{I}
\end{Highlighting}
\end{Shaded}

\begin{verbatim}
##      [,1] [,2] [,3]
## [1,]    1    0    0
## [2,]    0    1    0
## [3,]    0    0    1
\end{verbatim}

Observe que \(a_{ij} =0\) para \(i \neq j\) e \(a_{ij} = 1\) para \(i = j\).

A \textbf{Matriz nula} tem todos os elementos iguais a 0. Por exemplo

\begin{equation*}
B = 
\begin{bmatrix}
0 & 0 & 0\\
0 & 0 & 0
\end{bmatrix}
\end{equation*}

é gerada pela função \texttt{matrix()} com o valor 0 (zero) sendo o único para preencher todas as possições da matriz.

\begin{Shaded}
\begin{Highlighting}[]
\NormalTok{B \textless{}{-}}\StringTok{ }\KeywordTok{matrix}\NormalTok{ (}\DecValTok{0}\NormalTok{, }\DataTypeTok{nrow=}\DecValTok{2}\NormalTok{, }\DataTypeTok{ncol=}\DecValTok{3}\NormalTok{)}
\NormalTok{B}
\end{Highlighting}
\end{Shaded}

\begin{verbatim}
##      [,1] [,2] [,3]
## [1,]    0    0    0
## [2,]    0    0    0
\end{verbatim}

\hypertarget{transposta-de-uma-matriz}{%
\section{Transposta de uma matriz}\label{transposta-de-uma-matriz}}

\textbf{Matriz transposta} é o resultado da troca das linhas pelas colunas e denotada por \(A^T\) ou \(A^\prime\).

Para

\begin{equation*}
A = 
\begin{bmatrix}
2 & 7 \\
0 & 1 \\
2 & 0 
\end{bmatrix}
\end{equation*}

a sua transposta \(A^\prime\) é

\begin{equation*}
A^\prime = 
\begin{bmatrix}
2 & 0 & 2\\
7 & 1 & 0
\end{bmatrix}
\end{equation*}

A função \texttt{t()} gera a matriz transposta de uma matriz. Observe o códio seguinte

\begin{Shaded}
\begin{Highlighting}[]
\NormalTok{aux \textless{}{-}}\StringTok{ }\KeywordTok{c}\NormalTok{(}\DecValTok{2}\NormalTok{,}\DecValTok{0}\NormalTok{,}\DecValTok{2}\NormalTok{,}\DecValTok{7}\NormalTok{,}\DecValTok{1}\NormalTok{,}\DecValTok{0}\NormalTok{)}

\NormalTok{A \textless{}{-}}\StringTok{ }\KeywordTok{matrix}\NormalTok{ (aux,}\DataTypeTok{nrow=}\DecValTok{3}\NormalTok{,}\DataTypeTok{ncol=}\DecValTok{2}\NormalTok{)}

\NormalTok{tranposta\_A \textless{}{-}}\StringTok{ }\KeywordTok{t}\NormalTok{(A)}

\NormalTok{A}
\end{Highlighting}
\end{Shaded}

\begin{verbatim}
##      [,1] [,2]
## [1,]    2    7
## [2,]    0    1
## [3,]    2    0
\end{verbatim}

\begin{Shaded}
\begin{Highlighting}[]
\NormalTok{tranposta\_A}
\end{Highlighting}
\end{Shaded}

\begin{verbatim}
##      [,1] [,2] [,3]
## [1,]    2    0    2
## [2,]    7    1    0
\end{verbatim}

\hypertarget{trauxe7o-de-uma-matriz}{%
\section{Traço de uma matriz}\label{trauxe7o-de-uma-matriz}}

\begin{itemize}
\tightlist
\item
  \textbf{Traço de matriz quadrada} é a soma dos elementos da diagonal principal de uma matriz:
\end{itemize}

\begin{equation*}
Traço(A)= \sum_{i,j} a_{ij} \quad \forall i = j
\end{equation*}

Para

\begin{equation*}
A = 
\begin{bmatrix}
7 & 2 & 3\\
2 & 8 & 4\\
3 & 4 & 9
\end{bmatrix}
\end{equation*}

os elementos da \textbf{diagonal principal} são 7, 8 e 9, logo:

\begin{equation*}
Traço(A)= 7 + 8 + 9 = 24
\end{equation*}

O traço de A pode ser obtido por meio das funções \texttt{diag()} e \texttt{sum()} como abaixo.

\begin{Shaded}
\begin{Highlighting}[]
\NormalTok{aux\textless{}{-}}\KeywordTok{c}\NormalTok{(}\DecValTok{7}\NormalTok{,}\DecValTok{2}\NormalTok{,}\DecValTok{3}\NormalTok{,}\DecValTok{2}\NormalTok{,}\DecValTok{8}\NormalTok{,}\DecValTok{4}\NormalTok{,}\DecValTok{3}\NormalTok{,}\DecValTok{4}\NormalTok{,}\DecValTok{9}\NormalTok{)}
\NormalTok{A\textless{}{-}}\StringTok{ }\KeywordTok{matrix}\NormalTok{(aux,}\DecValTok{3}\NormalTok{,}\DecValTok{3}\NormalTok{)}
\KeywordTok{sum}\NormalTok{(}\KeywordTok{diag}\NormalTok{(A))}
\end{Highlighting}
\end{Shaded}

\begin{verbatim}
## [1] 24
\end{verbatim}

\hypertarget{determinante-de-uma-matriz}{%
\section{Determinante de uma matriz}\label{determinante-de-uma-matriz}}

O \textbf{Determinante de matriz} é um valor associado a uma matriz quadrada, A, e denotada por \(|A|\). A função \texttt{det()} calcula o determinante de uma matriz.

\begin{Shaded}
\begin{Highlighting}[]
\NormalTok{aux\textless{}{-}}\KeywordTok{c}\NormalTok{(}\DecValTok{7}\NormalTok{,}\DecValTok{2}\NormalTok{,}\DecValTok{3}\NormalTok{,}\DecValTok{2}\NormalTok{,}\DecValTok{8}\NormalTok{,}\DecValTok{4}\NormalTok{,}\DecValTok{3}\NormalTok{,}\DecValTok{4}\NormalTok{,}\DecValTok{9}\NormalTok{)}
\NormalTok{A\textless{}{-}}\StringTok{ }\KeywordTok{matrix}\NormalTok{(aux,}\DecValTok{3}\NormalTok{,}\DecValTok{3}\NormalTok{)}
\KeywordTok{det}\NormalTok{(A)}
\end{Highlighting}
\end{Shaded}

\begin{verbatim}
## [1] 332
\end{verbatim}

A \textbf{Matriz singular} é a matriz A cujo determinante é nulo, \(|A| = 0\). A matriz B é singular

\begin{Shaded}
\begin{Highlighting}[]
\NormalTok{aux\textless{}{-}}\KeywordTok{c}\NormalTok{(}\DecValTok{7}\NormalTok{,}\DecValTok{2}\NormalTok{,}\DecValTok{3}\NormalTok{,}\DecValTok{14}\NormalTok{,}\DecValTok{4}\NormalTok{,}\DecValTok{6}\NormalTok{,}\DecValTok{3}\NormalTok{,}\DecValTok{4}\NormalTok{,}\DecValTok{9}\NormalTok{)}
\NormalTok{A\textless{}{-}}\StringTok{ }\KeywordTok{matrix}\NormalTok{(aux,}\DecValTok{3}\NormalTok{,}\DecValTok{3}\NormalTok{)}
\KeywordTok{det}\NormalTok{(A)}
\end{Highlighting}
\end{Shaded}

\begin{verbatim}
## [1] 0
\end{verbatim}

\textbf{Propriedades do determinante:}

\begin{itemize}
\tightlist
\item
  O determinante de uma matriz quadrada A é igual ao determinante da sua transposta: \(|A| = |A^\prime|\);
\item
  Caso exista uma linha ou coluna na matriz igual a zero, o determinante é zero;
\item
  Caso exista duas filas paralelas, iguais ou proporcional, o determinante é zero;
\item
  O determinante do produto de um número real k por uma matriz A é igual ao produto de k elevado a n, onde n é o número de linhas de A, pelo determinante de A: \(|k . A| = k^n . |A|\);
\item
  Caso os elementos abaixo ou acima da diagonal principal forem nulos, o determinante será o produto dos elementos da diagonal principal;
\item
  Teorema de Binet: Seja A e B matrizes quadradas de ordem n, o determinante do produto de A por B é igual ao produto dos determinantes de A e B: \(|AB|=|A|.|B|\).
\end{itemize}

\hypertarget{matriz-inversa}{%
\section{Matriz inversa}\label{matriz-inversa}}

A \textbf{Matriz inversa} de uma matriz \(A\) é representada por \(A^{-1}\) é aquela que

\begin{equation*}
AA^{-1}=A^{-1}A = I
\end{equation*}

onde \(I\) é a matriz identidade. Para uma matriz \(A\) ter inversa é necessário e suficiente que seja \textbf{quadrada} e \textbf{não sigular}, isto é, o determinante é diferente de zero.

Para

\begin{equation*}
A = 
\begin{bmatrix}
 2  & -5\\
-1  &  3
\end{bmatrix}
\end{equation*}

a sua matriz inversa é

\begin{equation*}
A^{-1} = 
\begin{bmatrix}
 3 & 5\\
 1 & 2
\end{bmatrix}
\end{equation*}

Verificando \(A^{-1}A=I\)

\begin{equation*}
\begin{bmatrix}
 2 & -5\\
-1 &  3
\end{bmatrix}
\begin{bmatrix}
 3 & 5\\
 1 & 2
\end{bmatrix}=
\begin{bmatrix}
1 & 0 \\
0 & 1 
\end{bmatrix}
\end{equation*}

A função \texttt{solve()} é usada para resulver sistemas de equações lineares e em uma de suas chamadas retornar a matriz inversa do seu argumento.

\begin{Shaded}
\begin{Highlighting}[]
\NormalTok{A =}\StringTok{ }\KeywordTok{matrix}\NormalTok{(}\KeywordTok{c}\NormalTok{(}\DecValTok{2}\NormalTok{,}\OperatorTok{{-}}\DecValTok{1}\NormalTok{,}\OperatorTok{{-}}\DecValTok{5}\NormalTok{,}\DecValTok{3}\NormalTok{),}\DecValTok{2}\NormalTok{,}\DecValTok{2}\NormalTok{)}
\NormalTok{inv\_A =}\StringTok{ }\KeywordTok{solve}\NormalTok{(A)}
\NormalTok{inv\_A}
\end{Highlighting}
\end{Shaded}

\begin{verbatim}
##      [,1] [,2]
## [1,]    3    5
## [2,]    1    2
\end{verbatim}

para verificar a relaçao entre a sua matriz e a suam matriz inversa

\begin{Shaded}
\begin{Highlighting}[]
\NormalTok{A }\OperatorTok{\%*\%}\StringTok{ }\NormalTok{inv\_A}
\end{Highlighting}
\end{Shaded}

\begin{verbatim}
##      [,1] [,2]
## [1,]    1    0
## [2,]    0    1
\end{verbatim}

\hypertarget{operauxe7uxf5es-com-matrizes-no-r}{%
\section{Operações com matrizes no R}\label{operauxe7uxf5es-com-matrizes-no-r}}

A \textbf{Soma ou subtração} de duas matrizes \textbf{com as mesmas dimensões} é obtida pela soma/subtração dos elementos correspondentes.

O operador para essas operações são \texttt{+} e \texttt{-}.

\begin{equation*}
\begin{bmatrix}
7 & 2 & 3\\
2 & 8 & 4\\
3 & 4 & 9
\end{bmatrix} + 
\begin{bmatrix}
3 & 8 & 7\\
8 & 2 & 6\\
7 & 6 & 1
\end{bmatrix} = 
\begin{bmatrix}
10 & 10 & 10\\
10 & 10 & 10\\
10 & 10 & 10
\end{bmatrix}
\end{equation*}

\begin{Shaded}
\begin{Highlighting}[]
\NormalTok{A \textless{}{-}}\StringTok{ }\KeywordTok{matrix}\NormalTok{ (}\KeywordTok{c}\NormalTok{(}\DecValTok{7}\NormalTok{,}\DecValTok{2}\NormalTok{,}\DecValTok{3}\NormalTok{,}\DecValTok{2}\NormalTok{,}\DecValTok{8}\NormalTok{,}\DecValTok{4}\NormalTok{,}\DecValTok{3}\NormalTok{,}\DecValTok{4}\NormalTok{,}\DecValTok{9}\NormalTok{),}\DecValTok{3}\NormalTok{,}\DecValTok{3}\NormalTok{)}
\NormalTok{B \textless{}{-}}\StringTok{ }\KeywordTok{matrix}\NormalTok{ (}\KeywordTok{c}\NormalTok{(}\DecValTok{3}\NormalTok{,}\DecValTok{8}\NormalTok{,}\DecValTok{7}\NormalTok{,}\DecValTok{8}\NormalTok{,}\DecValTok{2}\NormalTok{,}\DecValTok{6}\NormalTok{,}\DecValTok{7}\NormalTok{,}\DecValTok{6}\NormalTok{,}\DecValTok{1}\NormalTok{),}\DecValTok{3}\NormalTok{,}\DecValTok{3}\NormalTok{)}
\NormalTok{A}
\end{Highlighting}
\end{Shaded}

\begin{verbatim}
##      [,1] [,2] [,3]
## [1,]    7    2    3
## [2,]    2    8    4
## [3,]    3    4    9
\end{verbatim}

\begin{Shaded}
\begin{Highlighting}[]
\NormalTok{B}
\end{Highlighting}
\end{Shaded}

\begin{verbatim}
##      [,1] [,2] [,3]
## [1,]    3    8    7
## [2,]    8    2    6
## [3,]    7    6    1
\end{verbatim}

\begin{Shaded}
\begin{Highlighting}[]
\NormalTok{A}\OperatorTok{+}\NormalTok{B}
\end{Highlighting}
\end{Shaded}

\begin{verbatim}
##      [,1] [,2] [,3]
## [1,]   10   10   10
## [2,]   10   10   10
## [3,]   10   10   10
\end{verbatim}

O \textbf{Produto por escalar} de uma matriz A por um escalar c é obtida multiplicando cada elemento de A pelo valor de \(c \in \mathbf{R}\).

O operador para essa operação \texttt{*}.

\begin{equation*}
5 \times
\begin{bmatrix}
3 & 8 & 7\\
8 & 2 & 6\\
7 & 6 & 1
\end{bmatrix} = 
\begin{bmatrix}
15 & 40 & 35\\
40 & 10 & 30\\
35 & 30 & 5
\end{bmatrix}
\end{equation*}

\begin{Shaded}
\begin{Highlighting}[]
\NormalTok{A \textless{}{-}}\StringTok{ }\KeywordTok{matrix}\NormalTok{ (}\KeywordTok{c}\NormalTok{(}\DecValTok{3}\NormalTok{,}\DecValTok{8}\NormalTok{,}\DecValTok{7}\NormalTok{,}\DecValTok{8}\NormalTok{,}\DecValTok{2}\NormalTok{,}\DecValTok{6}\NormalTok{,}\DecValTok{7}\NormalTok{,}\DecValTok{6}\NormalTok{,}\DecValTok{1}\NormalTok{),}\DecValTok{3}\NormalTok{,}\DecValTok{3}\NormalTok{)}
\NormalTok{A}
\end{Highlighting}
\end{Shaded}

\begin{verbatim}
##      [,1] [,2] [,3]
## [1,]    3    8    7
## [2,]    8    2    6
## [3,]    7    6    1
\end{verbatim}

\begin{Shaded}
\begin{Highlighting}[]
\DecValTok{5}\OperatorTok{*}\NormalTok{A}
\end{Highlighting}
\end{Shaded}

\begin{verbatim}
##      [,1] [,2] [,3]
## [1,]   15   40   35
## [2,]   40   10   30
## [3,]   35   30    5
\end{verbatim}

Na situação quando duas matrizes A e B tem o mesmo número de linhas e colunas o operador \texttt{*} executa o produto elemento a elemento.

\begin{equation*}
\begin{bmatrix}
7 & 2 & 3\\
2 & 8 & 4\\
3 & 4 & 9
\end{bmatrix} \circ 
\begin{bmatrix}
3 & 8 & 7\\
8 & 2 & 6\\
7 & 6 & 1
\end{bmatrix} =
\begin{bmatrix}
21 & 16 & 21\\
16 & 64 & 24\\
21 & 24 & 9
\end{bmatrix}
\end{equation*}

No R seria

\begin{Shaded}
\begin{Highlighting}[]
\NormalTok{A \textless{}{-}}\StringTok{ }\KeywordTok{matrix}\NormalTok{(}\KeywordTok{c}\NormalTok{(}\DecValTok{7}\NormalTok{,}\DecValTok{2}\NormalTok{,}\DecValTok{3}\NormalTok{,}\DecValTok{2}\NormalTok{,}\DecValTok{8}\NormalTok{,}\DecValTok{2}\NormalTok{,}\DecValTok{3}\NormalTok{,}\DecValTok{4}\NormalTok{,}\DecValTok{8}\NormalTok{),}\DecValTok{3}\NormalTok{,}\DecValTok{3}\NormalTok{)}
\NormalTok{B \textless{}{-}}\StringTok{ }\KeywordTok{matrix}\NormalTok{(}\KeywordTok{c}\NormalTok{(}\DecValTok{3}\NormalTok{,}\DecValTok{8}\NormalTok{,}\DecValTok{7}\NormalTok{,}\DecValTok{8}\NormalTok{,}\DecValTok{2}\NormalTok{,}\DecValTok{6}\NormalTok{,}\DecValTok{7}\NormalTok{,}\DecValTok{6}\NormalTok{,}\DecValTok{1}\NormalTok{),}\DecValTok{3}\NormalTok{,}\DecValTok{3}\NormalTok{)}
\NormalTok{A}
\end{Highlighting}
\end{Shaded}

\begin{verbatim}
##      [,1] [,2] [,3]
## [1,]    7    2    3
## [2,]    2    8    4
## [3,]    3    2    8
\end{verbatim}

\begin{Shaded}
\begin{Highlighting}[]
\NormalTok{B}
\end{Highlighting}
\end{Shaded}

\begin{verbatim}
##      [,1] [,2] [,3]
## [1,]    3    8    7
## [2,]    8    2    6
## [3,]    7    6    1
\end{verbatim}

\begin{Shaded}
\begin{Highlighting}[]
\NormalTok{A}\OperatorTok{*}\NormalTok{B}
\end{Highlighting}
\end{Shaded}

\begin{verbatim}
##      [,1] [,2] [,3]
## [1,]   21   16   21
## [2,]   16   16   24
## [3,]   21   12    8
\end{verbatim}

O \textbf{Produto de matrizes} só pode ser realizada caso o número de colunas da matriz que pré-multiplica for igual ao número de linhas da matriz que pós-multiplica. A operação consiste na soma dos produtos de cada elementos da linha da matriz que pré-multiplica pelos respectivos elementos da coluna da matriz que pós-multiplica. O resultado será uma matriz com o número de linhas da que pré-multiplica e com número de colunas das que pós-multiplica.

O operador para essas operações são \texttt{\%*\%}.

\begin{equation*}
\begin{bmatrix}
7 & 2 \\
2 & 8 \\
\end{bmatrix} + 
\begin{bmatrix}
3 & 8 & 7\\
8 & 2 & 6
\end{bmatrix} = 
\begin{bmatrix}
37 & 60 & 61\\
70 & 80 & 62
\end{bmatrix}
\end{equation*}

\begin{Shaded}
\begin{Highlighting}[]
\NormalTok{A \textless{}{-}}\StringTok{ }\KeywordTok{matrix}\NormalTok{(}\KeywordTok{c}\NormalTok{(}\DecValTok{7}\NormalTok{,}\DecValTok{2}\NormalTok{,}\DecValTok{2}\NormalTok{,}\DecValTok{8}\NormalTok{),}\DecValTok{2}\NormalTok{,}\DecValTok{2}\NormalTok{)}
\NormalTok{B \textless{}{-}}\StringTok{ }\KeywordTok{matrix}\NormalTok{(}\KeywordTok{c}\NormalTok{(}\DecValTok{3}\NormalTok{,}\DecValTok{8}\NormalTok{,}\DecValTok{8}\NormalTok{,}\DecValTok{2}\NormalTok{,}\DecValTok{7}\NormalTok{,}\DecValTok{6}\NormalTok{),}\DataTypeTok{nrow =} \DecValTok{2}\NormalTok{, }\DataTypeTok{ncol =}\DecValTok{3}\NormalTok{)}
\NormalTok{A}
\end{Highlighting}
\end{Shaded}

\begin{verbatim}
##      [,1] [,2]
## [1,]    7    2
## [2,]    2    8
\end{verbatim}

\begin{Shaded}
\begin{Highlighting}[]
\NormalTok{B}
\end{Highlighting}
\end{Shaded}

\begin{verbatim}
##      [,1] [,2] [,3]
## [1,]    3    8    7
## [2,]    8    2    6
\end{verbatim}

\begin{Shaded}
\begin{Highlighting}[]
\NormalTok{A}\OperatorTok{\%*\%}\NormalTok{B}
\end{Highlighting}
\end{Shaded}

\begin{verbatim}
##      [,1] [,2] [,3]
## [1,]   37   60   61
## [2,]   70   32   62
\end{verbatim}

\hypertarget{regressuxe3o-linear-muxfaltipla}{%
\chapter{Regressão Linear Múltipla}\label{regressuxe3o-linear-muxfaltipla}}

\hypertarget{modelo-de-regressuxe3o-linear-muxfaltipla}{%
\section{Modelo de Regressão Linear Múltipla}\label{modelo-de-regressuxe3o-linear-muxfaltipla}}

Seja a relação linear entre uma variável dependente Y e p variáveis independentes X. Então, o modelo estatístico de uma regressão linear, nos parâmetros, múltipla com p variáveis independentes e um termo aleatório, \(\epsilon\), é dado por

\begin{equation*}
Y_i = \beta_0 + \beta_1X_{1i} + \beta_2X_{2i} + \cdots +
+ \beta_pX_{pi} + \epsilon_i
\end{equation*}

com \(i=1,2,\ldots,n\). De forma alternativa, tem-se

\begin{equation*}
Y_i = \beta_0 + \sum_{j=1}^p\beta_j X_{ji} + \epsilon_i
\end{equation*}.

Uma outra forma de expressar as relações, fazendo \(i=1,2,\ldots, n\), surgem as equações:

\begin{align}
Y_1 &= \beta_0 + \beta_1 X_{11} + \beta_2 X_{21} + \cdots + \beta_p X_{p1}\\
Y_2 &= \beta_0 + \beta_1 X_{12} + \beta_2 X_{22} + \cdots + \beta_p X_{p2}\\
Y_3 &= \beta_0 + \beta_1 X_{13} + \beta_2 X_{23} + \cdots + \beta_p X_{p3}\\
\vdots & \\
Y_n &= \beta_0 + \beta_1 X_{1n} + \beta_2 X_{2n} + \cdots + \beta_p X_{pn}
\end{align}

Em notação matricial, esse sistema de equações pode ser expressa por

\begin{equation*}
Y = X\beta + \epsilon
\end{equation*}

que de forma explícita é

\begin{equation*}
\begin{bmatrix}
Y_1\\
Y_2\\
\vdots \\
Y_n
\end{bmatrix}=
\begin{bmatrix}
1 & X_{11} & X_{12} & \cdots & X_{1p} \\
1 & X_{21} & X_{22} & \cdots & X_{2p} \\
\vdots &        & \cdots &            \\
1 & X_{n1} & X_{n2} & \cdots & X_{np} \\
\end{bmatrix}
\begin{bmatrix}
\beta_0\\
\beta_1\\
\beta_2\\
\vdots \\
\beta_p
\end{bmatrix}+
\begin{bmatrix}
\epsilon_0\\
\epsilon_1\\
\epsilon_2\\
\vdots \\
\epsilon_p
\end{bmatrix}
\end{equation*}

Sendo

\begin{itemize}
\tightlist
\item
  n o número de observações e p a quantidade de variáveis explicativas;
\item
  X uma matriz \(n \times (p+1)\);
\item
  Y um vetor \(n \times 1\);
\item
  \(\beta\) um vetor \((p+1) \times 1\); e
\item
  \(\epsilon\) um vetor \(n \times 1\).
\end{itemize}

O problema consiste em obter o modelo ajustado:

\begin{equation*}
\hat{Y}_i = \hat{\beta}_0 + \hat{\beta}_1X_{1i} + \hat{\beta}_2X_{2i} + \cdots +
+ \hat{\beta}_p X_{pi}
\end{equation*}

É, para isso, deve-se obter o vetor \(\beta\). Admita-se as seguintes pressuposições:

\begin{enumerate}
\def\labelenumi{\arabic{enumi}.}
\tightlist
\item
  A variável dependente \(Y\) é a função linear das variáveis independentes X.
\item
  Os valores das variáveis independentes são fixos.
\item
  A média dos erros é nula, isto é, \(E\left(\epsilon_i\right)=0\).
\item
  Os erros são homoscedásticos, assim, \(V\left(\epsilon_i\right)=\sigma^2\).
\item
  Os erros são não correlacionados entre si, isto é, \(E\left(\epsilon_i\epsilon_j\right)=0\), para \(i \ne j\).
\item
  Os erros têm distribuição normal.
\end{enumerate}

Considere algumas consequências:

\begin{itemize}
\item
  Combine (4) e (5) para \(E\left(\epsilon^\prime\epsilon\right)=I\sigma^2\).
\item
  As pressuposições (1), (2) e (3) são necessárias para demostrar que os estimadores de Mínimos Quadrados são \textbf{não tendenciosos}.
\item
  As pressuposições de (1) a (5) permitem demonstrar que tais estimadores são \textbf{não tendenciosos} e de \textbf{variância mínima}.
\item
  A pressuposição (6) é necessária para construção de teste de hipóteses e de intervalos de confiânça para os parãmetros.
\end{itemize}

\hypertarget{estimadores-dos-paruxe2metros}{%
\section{Estimadores dos parâmetros}\label{estimadores-dos-paruxe2metros}}

O Método dos Mínimos Quadrados consiste em adotar como estimativas dos parâmetros os valores que minimiza a soma de quadrados dos desvios.

O modelo \(Y = X\beta+\epsilon\) têm \(\epsilon = Y - X\beta\), então a soma dos quadrados dos desvios é dada por

\begin{align}
Z &= \epsilon^\prime\epsilon \\
  &= \left(Y-X\beta\right)^\prime\left(Y-X\beta\right)\\
  &= Y^\prime Y - Y^\prime X\beta - \beta^\prime X^\prime Y + \beta^\prime X^\prime X \beta
\end{align}

Como \(Y^\prime X\beta = \beta^\prime X^\prime Y\) são iguais, então

\begin{align}
Z &= Y^\prime Y -2 \beta^\prime X^\prime Y + \beta^\prime X^\prime X \beta
\end{align}

A função \(Z\) apresenta seu valor mínimo para \(\beta\) que tornem a diferencial de \(Z\) e igualar a 0 (zero).

\begin{equation*}
\dfrac{\partial Z}{\partial \beta}= -2\left(\partial\beta^\prime\right) X^\prime Y + \left(\partial\beta^\prime\right) X^\prime X\hat{\beta} + \hat{\beta}^\prime X^\prime X\left(\partial \beta\right) = 0
\end{equation*}

Uma vez que \(\left(\partial\beta^\prime\right)X^\prime X\hat{\beta} =\hat{\beta}^\prime X^\prime X\left(\partial \beta\right)\) tem-se

\begin{align}
\dfrac{\partial Z}{\partial \beta}&= -2\left(\partial\beta^\prime\right) X^\prime Y + 2\left(\partial\beta^\prime\right) X^\prime X\hat{\beta} \\
  &=2\left(\partial\beta^\prime\right)\left[X^\prime X\hat{\beta}  -X^\prime Y\right]
\end{align}

Assim, para definir \(\hat{\beta}\) faça

\begin{equation*}
X^\prime X\hat{\beta}-X^\prime Y = 0
\end{equation*}

Logo,
\begin{equation*}
\hat{\beta} = \left(X^\prime X\right)^{-1}X^\prime Y.
\end{equation*}

\hypertarget{methods}{%
\chapter{Methods}\label{methods}}

We describe our methods in this chapter.

\hypertarget{applications}{%
\chapter{Applications}\label{applications}}

Some \emph{significant} applications are demonstrated in this chapter.

\hypertarget{example-one}{%
\section{Example one}\label{example-one}}

\hypertarget{example-two}{%
\section{Example two}\label{example-two}}

\hypertarget{final-words}{%
\chapter{Final Words}\label{final-words}}

We have finished a nice book.

  \bibliography{book.bib,packages.bib}

\end{document}
